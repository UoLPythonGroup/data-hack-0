\documentclass[12pt]{article}
\usepackage[a4paper, total={7in, 9in}]{geometry}
\begin{document}
\thispagestyle{empty}

\Large{
\begin{center}
{\bf 2/7/14 Datahack-0: Geographical data}
\end{center}

\begin{enumerate}

\item {\bf Quantification}
\begin{enumerate}
\item What is the total length of cycle paths in Leeds?
\item What is the fastest/flattest/safest/best route from X to Y?
\item Where's my nearest `good' 10k cycle route?
\item What happens if we build a new route?
\item Can we add time/weather/seasonal hazards and effects?
\end{enumerate}

\item {\bf Network analysis}

\begin{enumerate}
\item How can junctions be detected?
\item How can we convert the information into a graph (e.g., networkx)?
\item Can we analysis properties of the network (e.g., centrality, communicability, persistent hubs, small-world property, etc)?
\item What is the network effect of building new roads?
\end{enumerate}

\item {\bf Computational}

\begin{enumerate}
\item How can we be confident in our answers?
\item Can we map our data onto OS maps?
\item How can we tell if data is automatically generated or actual input from users?
\item Is Python a good or bad tool for this type of problem?
\end{enumerate}

\item {\bf Big-data}


\begin{enumerate}
\item Which methods, algorithms and procedures scale well with geographical size?
\item Is anything we did more generally applicable?
\end{enumerate}

\end{enumerate}
}
\end{document}